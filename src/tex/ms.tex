% Define document class
\documentclass[modern]{aastex631}
\usepackage{showyourwork}

\newcommand{\dd}{\mathrm{d}}

% Begin!
\begin{document}

% Title
\title{How Many Samples Do We Need?}

% Author list
\author[0000-0003-1540-8562]{Will M. Farr}
\email{will.farr@stonybrook.edu}
\affiliation{Department of Physics and Astronomy, Stony Brook University, Stony Brook, NY 11794, USA}
\email{wfarr@flatironinstitute.org}
\affiliation{Center for Computational Astrophysics, Flatiron Institute, New York, NY 10010, USA}

% Abstract with filler text
\begin{abstract}
    Lorem ipsum dolor sit amet, consectetuer adipiscing elit.
    Ut purus elit, vestibulum ut, placerat ac, adipiscing vitae, felis.
    Curabitur dictum gravida mauris, consectetuer id, vulputate a, magna.
    Donec vehicula augue eu neque, morbi tristique senectus et netus et.
    Mauris ut leo, cras viverra metus rhoncus sem, nulla et lectus vestibulum.
    Phasellus eu tellus sit amet tortor gravida placerat.
    Integer sapien est, iaculis in, pretium quis, viverra ac, nunc.
    Praesent eget sem vel leo ultrices bibendum.
    Aenean faucibus, morbi dolor nulla, malesuada eu, pulvinar at, mollis ac.
    Curabitur auctor semper nulla donec varius orci eget risus.
    Duis nibh mi, congue eu, accumsan eleifend, sagittis quis, diam.
    Duis eget orci sit amet orci dignissim rutrum.
\end{abstract}

% Main body with filler text
\section{Introduction}
\label{sec:intro}

The form of the hierarchical marginal likelihood is 
\begin{equation}
    \log \mathcal{L}\left( \mathbf{d} \mid \lambda \right) = \sum_{i=1}^{N} \log \int \dd \theta_i \, p\left( d_i \mid \theta_i \right) p\left( \theta_i \mid \lambda \right).
\end{equation}
We often approximate the integrals in this expression using Monte Carlo
integration over samples for each event $i$, $\theta_{i}^{(s)}$, $s = 1, \ldots,
S_i$, drawn from a posterior density using some fiducial prior, $p\left( \theta
\right)$:
\begin{equation}
    \log \mathcal{L}\left( \mathbf{d} \mid \lambda \right) \approx \sum_{i=1}^{N} \log \frac{1}{S_i} \sum_{s=1}^{S_i} \frac{p\left( \theta_i^{(s)} \mid \lambda \right)}{p\left( \theta_i^{(s)} \right)} .
\end{equation}

\bibliography{bib}

\end{document}
